% simple.tex - A simple article to illustrate document structure.

% Andrew Roberts - June 2003

\documentclass[twocolumn]{article}
\usepackage{}
\usepackage{listings}

\begin{document}
\lstset{language=Python,showstringspaces=false}

% Article top matter
\title{Introduction to Programming Workshop - Python} 
\author{Thomas Mortensson and Ben Collins\\
        	Computer Science Department,\\
		University of Bristol,\\
		\texttt{tm0797@bristol.ac.uk, 
		bc0517@bristol.ac.uk}}  %\texttt formats the text to a typewriter style font
\date{\today}  %\today is replaced with the current date
\maketitle

\begin{abstract}
This tutorial focused class will introduce participants to the core concepts of Python on a Raspberry Pi. Programming learned in this session is transferrable to many different platforms and it's use is not limited to the Raspberry Pi Hardware.\\

 Requirements for this programming session are  few as the session is designed to be workable on any machine with Python 2.x installed.
\end{abstract}

\section{Introduction to Python - What is it?}

\section {Programming Python}



\subsection{Python shell}

\subsection{Hello World!}

In this first example we will build a simple hello world application. The purpose of building this application is to familiarise a new programmer with the python programming interface. The Hello world program simply outputs the words ``Hello World!" to the computers display. We generate the hello world program by typing the following into the Python shell: 


\begin{lstlisting}

print "Hello World!"

\end{lstlisting}


\subsection{Variables and Assignment}

\subsection{Types}

\subsection{Producing out first source file!}

\subsection{User Input}

\subsection{Exercise 1}

\subsection{IF Statements and Boolean Logic}

\subsection{Loops}

\subsection{Exercise 2}

\subsection{Functions}

\subsection{Exercise 3}

\section{Further Reading}

\subsection{Raspberry Pi Hardware with Python - GPIO}



\end{document}  %End of document.
