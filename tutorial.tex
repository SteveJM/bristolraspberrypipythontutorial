% Thomas Mortensson - September 2013

\documentclass[twocolumn]{article}
\usepackage[margin=0.5in]{geometry}
\usepackage{listings}
\usepackage{hyperref}
\usepackage[usenames,dvipsnames]{color}
\usepackage{booktabs}

\renewcommand{\familydefault}{\sfdefault}
\definecolor{DarkGreen}{rgb}{0.0,0.4,0.0} % Comment color
\definecolor{highlight}{RGB}{255,251,204} % Code highlight color

\lstdefinestyle{Style1}{ % Define a style for your code snippet, multiple definitions can be made if, for example, you wish to insert multiple code snippets using different programming languages into one document
language=Python, % Detects keywords, comments, strings, functions, etc for the language specified
backgroundcolor=\color{highlight}, % Set the background color for the snippet - useful for highlighting
basicstyle=\footnotesize\ttfamily, % The default font size and style of the code
breakatwhitespace=false, % If true, only allows line breaks at white space
breaklines=true, % Automatic line breaking (prevents code from protruding outside the box)
captionpos=b, % Sets the caption position: b for bottom; t for top
commentstyle=\usefont{T1}{pcr}{m}{sl}\color{DarkGreen}, % Style of comments within the code - dark green courier font
deletekeywords={}, % If you want to delete any keywords from the current language separate them by commas
%escapeinside={\%}, % This allows you to escape to LaTeX using the character in the bracket
firstnumber=1, % Line numbers begin at line 1
frame=single, % Frame around the code box, value can be: none, leftline, topline, bottomline, lines, single, shadowbox
frameround=tttt, % Rounds the corners of the frame for the top left, top right, bottom left and bottom right positions
keywordstyle=\color{Blue}\bf, % Functions are bold and blue
morekeywords={}, % Add any functions no included by default here separated by commas
numbers=left, % Location of line numbers, can take the values of: none, left, right
numbersep=10pt, % Distance of line numbers from the code box
numberstyle=\tiny\color{Gray}, % Style used for line numbers
rulecolor=\color{black}, % Frame border color
showstringspaces=false, % Don't put marks in string spaces
showtabs=false, % Display tabs in the code as lines
stepnumber=5, % The step distance between line numbers, i.e. how often will lines be numbered
stringstyle=\color{Purple}, % Strings are purple
tabsize=2, % Number of spaces per tab in the code
}

\begin{document}
\lstset{style=Style1}

\title{Introduction to Programming Workshop --- Python} 
\author{Thomas Mortensson and Ben Collins\\
        	Computer Science Department,\\
		University of Bristol,\\
		\texttt{tm0797@bristol.ac.uk, 
		bc0517@bristol.ac.uk}} 
\date{\today} 
\maketitle

\section{Introduction to Python - What is it?}

Python is a high-level programming languague whith a main design goal of code readability. For this purpose it is an excellent introductory porogramming languague as it allows programs to be written in fewer lines of code, faster and in a way that is generally readable to even non programmers. Nowadays many operating systems are shipped with Python built in. Mac OS X and most versions of Linux (Including debian which is the operating system the Raspberry Pi runs) include Python by default. Python is often used for rapid prototyping due to it's large array of libraries and can be used to control many diverse applications such as robotics and next-generation web services. 

\section {Programming Python}



\subsection{Python shell}

The Python Shell is a brilliant tool allowing us to interactively start programming. What this means is that we can type commands and get instant results, allowing us to see what each line of our program does. To open the Python Shell look on the Raspberry Pi Desktop for an icon saying LXTerminal and click that. Once that opens you should be presented with a black box which says:
\begin{lstlisting}
pi@raspberrypi ~ $
\end{lstlisting}
what you need to do is type in ``python'' and hit enter. You should then be presented with:
\begin{lstlisting}
pi@raspberrypi ~ $ python
Python 2.7.3
>>>
\end{lstlisting}
Congratulations, you're there!


\subsection{Hello World!}

In this first example we will build a simple Hello World application. The purpose of building this application is to familiarise a new programmer with how to write a simple program in Python. The Hello World program simply outputs the words ``Hello World!" to the computer's display. We generate the Hello World program by typing the following into the Python shell:\\

\begin{lstlisting}
print "Hello World!"
\end{lstlisting}
After pressing Enter to execute the line in Python Shell the computer should respond with a line printed on the standard output: Hello World!\\ The quotation marks around the words signify that this portion of data is a string. More on this in the Types section. You can modify this program to write anything to the screen.


\subsection{Variables and Assignment}

In the previous section we completed the most basic starter program you will encounter in any programming language. We will now extend this by printing the sum of two numbers. Below is some example code to try:

\begin{lstlisting}
print 2 + 5
\end{lstlisting}
Can you guess what this will print to the screen? Try running this in your Python Shell, This will not produce the output ``2 + 5''. Instead as we have removed the quotation marks from the input, Python instead or reading the characters we've typed as a string now interprets the characters as two numbers with an addition symbol. Python will execute this expression and return the correct result of 7. \\
\\
If we take this a step further we can use something called a variable to store the result of our calculation before we print it. This is shown in the below code:

 \begin{lstlisting}
i = 2 + 5
print i
\end{lstlisting}
We have created a variable called ``i'' and we set its value to be 7; we then print the value of i (which is 7). Note that we could have named the variable anything - we just chose ``i'' since it is a traditional name for a variable.\\
\\
Try the following code:
\begin{lstlisting}
i = 2 + 5
i = i * 3
print i
\end{lstlisting}
Write here what you expect the program to print to the screen:\_\_\_\\
\\
Now run this program in your Python Shell, what is the result?\\
\\
When we calculate expressions we must be careful to obey the laws of BIDMAS, more on this here:\\
\\ \url{http://www.bbc.co.uk/bitesize/ks3/maths/number/order\_operation/revision/2/}\\
For Example:
\begin{lstlisting}
i = 2 + 1
i = i * 6
i = i / 3
i = i - 1
print i

j = 2 + 1 * 6 / 3 - 1
print j

k = ((((2 + 1) * 6) / 3) - 1)
print k
\end{lstlisting}
This program shows the different outputs obtained by using and not using the rules of BIDMAS.\\
\\
Another form of expression we will cover is called String concatenation. To concatenate two Strings means to join them together. For example if I wanted to concatenate ``Hello '' and ``World!'' into ``Hello World!'' we use a comma between our two strings as demonstrated in this program
\begin{lstlisting}
name = "World!"
print "Hello ", name
\end{lstlisting}
\subsection{Types}
As with all programming languagues, Python will make use of different types to store different pieces of data. Examples of basic types are Integer, Float, String and Boolean. An Integer is any whole number. For example the number 3 is an integer whereas 3.1 is not. A floating point number (or Float) is a number which includes a decimal point. Examples of a float are -3.1 or 5.0 (beware of the difference between 5.0 and 5). A String is used to define text, we have used strings previously to define words in our Hello World program. Strings are denoted by the quotation marks that surround them, e.g. ``Hello'' is a String but Hello is not. Python automatically assigns types to variables when they are used based on what data you have stored in them. We can see what data type Python has assigned by using the function type() (More on functions later on!). This is demonstrated in this example:
\begin{lstlisting}
integer = 5
float_number = 1.2
string = "Ben"
boolean = True
integer2 = int("3")

print integer, type(integer)
print float_number, type(float_number)
print string, type(string)
print boolean, type(boolean)
print integer2, type(integer2)
\end{lstlisting}
This should print out the types of all the variables you have defined! Note: we are using a system called ``typecasting'' to define integer2, this allows us to explicitly force the computer to use a specific datatype for a variable.
\subsection{Producing our first source file!}
So far in our Python experience we have been running all of our programs interactively in the Python Shell. While this may be good for building quick programs, the vast majority of the time as a programmer you will want to save your work. To build a Python source file we write the commands we have been writing in the shell directly into a blank text file. Once you have written the commands you wish to execute you have to save your program as your\_program\_name\_here.py . I recommend you save your source files in the home directory (This may be called pi on your Raspberry Pi) and execute them by opening a Linux Terminal and typing:
\begin{lstlisting}
python your_program_name_here.py
\end{lstlisting}
Try creating a source file with your Hello World program in and running it.
\subsection{User Input}
Many times when you are producing a program in Python you will need to take some form of input from the user. This is done in Python using the raw\_input function:
\begin{lstlisting}
name = raw_input("Enter your name: ") 
print "Hello", name
\end{lstlisting}
The function raw\_input() takes data written to the standard input (In this case the Linux terminal's text input) and stores it in a local variable. We are using a variable called name in this instance. The raw\_input() function will store data in what is deemed the relevant datatype. For example if I write a name, type(name) will evaluate to a String. If 3.0 is entered a type(name) will evaluate to float etc.

\subsection{IF Statements and Boolean Logic}

So far in our programming tutorial we have created programs that simply run whatever you have written. However, in a typical program we may not want to run all lines, but work out which lines should be run depending on what the user does. All programs will have an Input, Process and Output stage. Without a form of flow control we will always be limited in what we can build. As this is the case we will now introduce the concept of an IF statement. An IF statement uses a comparison to control the flow of a program.\\
\\
In Computer Science information is processed as Binary - 0's and 1's. These 0's and 1's represent the fundamental building blocks of how a computer operates. We will not go too deeply into how Binary works - however, note that a 1 can represent the Boolean value True and the value 0 can represent the Boolean value False. An IF statement allows a computer to execute a comparison based upon the conditions set by a programmer. For example the statement ``3 $>$ 2'' will evaluate to True as the number 3 is larger than the number 2. If we use an IF statement we can choose to begin a branch of execution based upon this result:
\begin{lstlisting}
if(3>2):
	print "3 is greater than 2"
\end{lstlisting}

We can extend this concept further by also having an else clause, a clause that is executed when the condition tested does not hold true. This is demonstrated in the next piece of code:
\begin{lstlisting}
works = True # Alternate this value
if works:
	print "This Works!"
else:
	print "This Doesn't Work!"
\end{lstlisting}
You can change the value of True in your source file from True to False to alter the program's output.

A core concept governing how we use IF statements is boolean logic. The 3 main operators as a programmer you will be required to know are how to use AND, OR and NOT operators. These operators work according to these 3 definitions:
\begin{itemize}
	\item AND --- returns True if both x and y are true, otherwise False. 
	\item OR --- returns True if x is True, or if y is True, or if both x and y are True. Otherwise False.
	\item NOT --- returns True if x is False and False if x is True.
\end{itemize}
This can be seen in the following Truth Tables --- You don't need to remember this, just keep it as a reference.

\begin{center}
	\begin{tabular}{lll} 
		\toprule
		x & y & x AND y \\
		\midrule
		False & False & False \\
		False & True & False \\
		True & False & False \\
		True & True & True \\
		\bottomrule
	\end{tabular}
	\quad
	\begin{tabular}{lll} 
		\toprule
		x & y & x OR y \\
		\midrule
		False & False & False \\
		False & True & True \\
		True & False & True \\
		True & True & True \\
		\bottomrule
	\end{tabular}
\end{center}

\begin{center}
\begin{tabular}{ll} 
		\toprule
		x & NOT x \\
		\midrule
		False & True \\
		True & False \\
		\bottomrule
	\end{tabular}
\end{center}
Try the following code by changing the values of ``apple'' and ``orange'' from True and False to see the effect this has on the and or and not operations.
\begin{lstlisting}
apple = True
orange = False
if apple and orange:
	print "Have both fruits"
if not apple:
	print "Don't have an apple"
if apple or orange:
	print "Have a fruit"
\end{lstlisting}
Once we understand AND, OR and NOT, we can progress to comparison operators. Comparison operators are extremely useful for comparing numbers in IF statements. There are 5 comparisons you will need to be aware of, these are:
\begin{center}
	\begin{tabular}{lll} 
		\toprule
		Maths & Programming & Description\\
		\midrule
		$<$ & $<$ & Less Than\\
		$\le$ & $<=$ & Less Than or Equal\\
		$=$ & $==$ & Equal\\
		$\ge$ & $>=$ & Greater Than or Equal\\
		$>$ & $>$ & Greater Than\\
		\bottomrule
	\end{tabular}
\end{center}
We can test these operators in the following interactive program:
\begin{lstlisting}
below_test = 5
above_test = 10
equals_test = 8

current_num = int(raw_input("Num: "))

if current_num < below_test:
	print 1
if current_num <= below_test:
	print 2
if current_num == equals_test:
	print 3
if current_num >= above_test:
	print 4
if current_num > above_test:
	print 5
\end{lstlisting}
We can see that by using different numbers we can get different return values. Try setting the variable ``current\_num'' to the values 3, 5, 6, 8, 10 and 11 and see if you can predict what numbers the program will return.

\subsection{Loops}
We have now covered many base concepts such as variables, expressions, operators, input and output as well as IF statements and Boolean logic. We will now build upon these concepts to introduce the concept of a loop. Computers are useful to us not only due to what they can process but also due to the fact that we can program them. Computers are a perfect solution in situations where a human would take months or years to complete a particular repetitive task. To repeat a task (or iterate) we require the use of a construct called a LOOP.\\
\\
Here is your first loop:
\begin{lstlisting}
for i in xrange(0, 10):
	print i
\end{lstlisting}
This will print out the first 10 numbers starting from 0 and counting up to 9. In Computer Science (and most programming languages) counting is done from 0.\\
\\
A loop is useful however its usefulness is greatly increased by using some form of comparison within the loop body. This is demonstrated in this code:
\begin{lstlisting}
below_test = 4
above_test = 7

for i in xrange(0,10):
	if i < below_test:
		print i, "- BELOW"
	if i > above_test:
		print i, "- ABOVE"
\end{lstlisting}
Here we print out BELOW when the variable ``i'' is less than 4 and ABOVE when it is greater than 7.\\
\\
We can add a new operator, modulo, into the mix to help us demonstrate the usage of AND, OR and NOT in a loop in our next example. Modulo works by working out the remainder if a number is divided by another number. For example 3 \% 2 = 1 because 3 divided by 2 = 2 remainder 1. We can use modulo and check for a remainder of 0 to check if a number is cleanly divisible by another number. You will see what we mean in this next example:
 \begin{lstlisting}
for i in xrange(0,50):
#	if (i % 2 == 0 or i % 5 == 0):
#	if (i % 2 == 0 and i % 5 == 0):
#	if (not i % 2 == 0):
		print i
\end{lstlisting}
In this piece of code we have also introduced comments. A comment is a way of writing a line of code and not executing it, Typically comments are used to document code and make it more readable however we can use them in this instance to turn lines of execution on or off. To comment out a line we add a Hash symbol (\#) to the beginning of the line. This will mean that the computer skips that line. When you execute this program as it is, it will first output the numbers from 0 to 49. If we remove the comment from the 2nd line this program will output all numbers between 0 and 49 which are even or divisible by 5. If we re-comment the 2nd line and uncomment the 3rd line the program will now output all numbers between 0 and 49 which are even and divisible by 5. This will therefore only print out multiples of 10. If we comment the 3rd line again and un-comment the 4th line we will now print all numbers between 0 and 49 which are not even - i.e. all of the odd numbers. Try this for yourself by changing where the comments are or the numbers the program is checking for divisibility.

\subsection{Functions}

The last core concept we will introduce in this document is the concept of functions. So far there has been a bit of repetition in our code - if we wanted to add numbers we had to do this manually each time. With a function call we can split apart different sections of our program. This helps us remove repetition, and allows us to make single large changes very quickly. For example if I build a function that's sole purpose is to add two numbers, I can then use this function in my code. If later I decide I want to instead subtract those 2 numbers in all cases where I have used my function I can change that line in one place rather than having to change every addition symbol to a subtraction symbol (and potentially forgetting some and introducing errors into my code). Functions allow us to logically divide a program to allow for more complex execution. In the simple examples we provide you may see less of a use for functions - however, when you have chunks of code 30 lines long which are similar you will want to start using functions to simplify your work (and save yourself a lot of time!). We have briefly touched on some basic functions likely without you realising. When we write a function we denote it by writing its name and putting brackets at the end of its name. If the function takes an input, we put that input between the brackets. We have already used the type() function. e.g ``print integer, type(integer)'' in the section Types and we have used the raw\_input() function in the Variables section. In the following example we will define a function print\_hello() (which takes 0 inputs) and we will run it in a loop.
\begin{lstlisting}
def print_hello():
	print "Hello World"

for i in xrange(0, 10):
	print_hello()
\end{lstlisting}
We can see that we have offloaded the work of printing hello world to a function rather than calling ``print \"Hello World!\"'' from within our loop.\\
\\
Expanding from this we can now create a function add() which can take 2 input parameters, we will have each parameter being a number to add. This is shown in the next example:
\begin{lstlisting}
def add(num1, num2):
	result = num1 + num2
	return result

res = 0
for i in xrange(0, 10):
	new_res = add(res, i)
	print res, "+", i, "=", new_res
	res = new_res
\end{lstlisting}
See if you can guess the program flow before running the program. Write on a piece of paper what you think the program will output before running it. This is a simple example of how we can use functions to offload work to the computer. Using a combination of a function and a for loop we have saved us having to write 10 seperate lines of addition. You can see how this can only help more in longer programs.


\section{Further Reading}

\subsection{Raspberry Pi Hardware with Python - GPIO}

In this last additional section I will show you how to do some extremely basic hardware interaction using the concepts of Python we have learnt in the tutorial sections. A brilliant website to show exactly what is going on here in more depth is: \url{http://code.google.com/p/raspberry-gpio-python/wiki/BasicUsage}
\begin{lstlisting}
import RPi.GPIO as GPIO
import time

GPIO.setmode(GPIO.BOARD)
GPIO.setup(16, GPIO.OUT)
GPIO.output(16, 0)
off = False
while True:
        if(off):
                GPIO.output(16,1)
                off = False
        else:
                GPIO.output(16,0)
                off = True
        time.sleep(1)
\end{lstlisting}
The first 2 lines, the import lines, tell Python to load some libraries it has stored (additional code). We use RPi.GPIO to interact with the GPIO pins and time to cause a delay with time.sleep(1).\\
\\
In the next 3 lines we have to setup the GPIO pins to allow us to use them, the first line tells the program what configuration the pins are in (Pinouts can be found here /url{http://www.hobbytronics.co.uk/raspberry-pi-gpio-pinout}), the second line tells the program to set PIN16 as an output and the 3rd line tells us to set the initial value of PIN16 to 0 (OFF). We then have a loop which cycles PIN16 from OFF to ON and back with a time delay of 1 second. We use a Boolean ``off'' to keep track of the state of the LED light so we know to turn it on or off within our loop. The overall output of the program should be a blinking LED if you attach an LED correctly to PIN16 and GROUND.



\end{document} 
